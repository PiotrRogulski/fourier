% !TeX encoding = UTF-8
% !TeX spellcheck = pl_PL

\PassOptionsToPackage{breaklinks}{hyperref}
\PassOptionsToPackage{svgnames}{xcolor}

\documentclass[polish, 12pt, aspectratio=169]{beamer}

\usepackage[nosingleletter, lastparline]{impnattypo}
\usepackage[polish]{babel}
\usepackage{minted}
\usepackage{csquotes}
\usepackage[backref, backend=bibtex]{biblatex}
\usepackage{bookmark}
\usepackage{hyperref}
\usepackage[babel, tracking]{microtype}
\usepackage{booktabs}
\usepackage{graphicx}
\usepackage{parskip}
\usepackage{framed}
\usepackage{tabularx}
\usepackage{ltablex}
\usepackage{adjustbox}
\usepackage{float}
\usepackage{longtable}
\usepackage{subcaption}
\usepackage[strings]{underscore}
\usepackage{tikz}
\usepackage{amsmath}
\usepackage{amssymb}

\usetheme[progressbar=frametitle]{metropolis}
\setmonofont{JetBrainsMono}[
    Scale=0.8,
    Extension=.ttf,
    UprightFont=*-Regular,
    BoldFont=*-Bold,
    ItalicFont=*-Italic,
    BoldItalicFont=*-BoldItalic
]

\title{Transformata Fouriera}
\author{Piotr Rogulski}
\date{\today}

\addbibresource{fourier.bib}

\begin{document}

\frame{\titlepage}

\begin{frame}{Agenda}
    \setbeamertemplate{section in toc}[sections numbered]
    \tableofcontents
\end{frame}

\section{Transformacja Fouriera}

\begin{frame}{Foo}
foo
\end{frame}

\section[DFT \\ {\normalsize Discrete Fourier Transform}]{DFT}

\begin{frame}{Bar}
bar
\end{frame}

\section[FFT \\ {\normalsize Fast Fourier Transform}]{FFT}

\begin{frame}{Baz}
baz
\end{frame}

\section{Zastosowania}

\begin{frame}{Qux}
qux
\end{frame}

\section*{Bibliografia}

\begin{frame}[allowframebreaks]{Bibliografia}
    \nocite{*}
    \printbibliography[heading=none]
\end{frame}

\end{document}
