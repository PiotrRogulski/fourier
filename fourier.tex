% !TeX encoding = UTF-8
% !TeX spellcheck = pl_PL

\PassOptionsToPackage{breaklinks}{hyperref}
\PassOptionsToPackage{svgnames}{xcolor}

\documentclass[polish, 12pt, aspectratio=169]{beamer}

\usepackage[nosingleletter, lastparline]{impnattypo}
\usepackage[polish]{babel}
\usepackage{minted}
\usepackage{csquotes}
\usepackage[backref, backend=bibtex]{biblatex}
\usepackage{bookmark}
\usepackage{hyperref}
\usepackage[babel, tracking]{microtype}
\usepackage{booktabs}
\usepackage{graphicx}
\usepackage{parskip}
\usepackage{framed}
\usepackage{tabularx}
\usepackage{ltablex}
\usepackage{adjustbox}
\usepackage{float}
\usepackage{longtable}
\usepackage{subcaption}
\usepackage[strings]{underscore}
\usepackage{tikz}
\usepackage{amsmath}
\usepackage{amssymb}
\usepackage{xfrac,unicode-math}
\usepackage{bm}

\usetheme[progressbar=frametitle]{metropolis}
\setmonofont{JetBrainsMono}[
    Scale=0.8,
    Extension=.ttf,
    UprightFont=*-Regular,
    BoldFont=*-Bold,
    ItalicFont=*-Italic,
    BoldItalicFont=*-BoldItalic
]

\title{Transformacja Fouriera}
\author{Piotr Rogulski}
\date{\today}

\addbibresource{fourier.bib}

\begin{document}

\frame{\titlepage}

\begin{frame}{Agenda}
    \setbeamertemplate{section in toc}[sections numbered]
    \tableofcontents
\end{frame}

\section{Transformacja Fouriera}

\begin{frame}

\end{frame}

\begin{frame}{Transformacja Fouriera}
    \Huge
    \begin{equation*}
        \mathcal{F}(f)(\xi) = \int_{-\infty}^{\infty} f(x) e^{-2\pi i x \xi} dx
    \end{equation*}
    \pause{}
    \small
    W postaci n-wymiarowej:
    \normalsize
    \vspace{-1em}
    \begin{equation*}
        \mathcal{F}(f)(\symbf{\xi}) = \int_{\mathbb{R}^n} f(\symbf{x}) e^{-2\pi i (\symbf{x \cdot \xi})} d\symbf{x}
    \end{equation*}
\end{frame}

\section[DFT \\ {\normalsize Discrete Fourier Transform}]{DFT}

\begin{frame}{Dyskretna transformacja Fouriera}
    \Huge
    \begin{equation*}
        X_k = \sum_{n=0}^{N-1} x_n e^{-2\pi i \frac{kn}{N}}
    \end{equation*}
    \pause{}
    \small
    W postaci n-wymiarowej:
    \normalsize
    \vspace{-1em}
    \begin{equation*}
        X_{\symbf{k}} = \sum_{\symbf{n} = \symbf{0}}^{\symbf{N} - \symbf{1}} x_{\symbf{n}} e^{-2\pi i \frac{\symbf{k \cdot n}}{\symbf{N}}}
    \end{equation*}
\end{frame}

\section[FFT \\ {\normalsize Fast Fourier Transform}]{FFT}

\begin{frame}{Baz}
baz
\end{frame}

\section{Zastosowania}

\begin{frame}{Qux}
qux
\end{frame}

\section*{Bibliografia}

\begin{frame}[allowframebreaks]{Bibliografia}
    \nocite{*}
    \printbibliography[heading=none]
\end{frame}

\end{document}
