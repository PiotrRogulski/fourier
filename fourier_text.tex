% !TeX program = lualatex
% !TeX encoding = UTF-8
% !TeX spellcheck = pl_PL
\documentclass[polish, titlepage, 12pt]{article}

\usepackage[svgnames]{xcolor}
\usepackage[nosingleletter, lastparline]{impnattypo}
\usepackage[polish]{babel}
\usepackage{csquotes}
\usepackage[breaklinks]{hyperref}
\usepackage{bookmark}
\usepackage[babel, tracking]{microtype}
\usepackage{booktabs}
\usepackage[margin=1in]{geometry}
\usepackage{graphicx}
\usepackage{parskip}
\usepackage{framed}
\usepackage{tabularx}
\usepackage{ltablex}
\usepackage{adjustbox}
\usepackage{float}
\usepackage{longtable}
\usepackage{subcaption}
\usepackage[strings]{underscore}
\usepackage{tikz}
\usepackage{amsmath}
\usepackage{xfrac,unicode-math}

\linespread{1.3}

\begin{document}

\section*{Historia}

\paragraph{Jean-Baptiste Joseph Fourier (1768--1830)}
\begin{itemize}
    \item Znaczący wpływ na rozwój badań nad przepływem ciepła \\
          Prawo Fouriera: \( \symbf{q} = -k \nabla T \)
    \item Wprowadził szereg Fouriera w celu rozwiązania równania ciepła
\end{itemize}

\paragraph{Siméon Denis Poisson (1781--1840)}
\begin{itemize}
    \item Zajmował się matematycznymi aspektami przewodnictwa ciepła
    \item Wprowadził pojęcie całki Poissona
\end{itemize}

\paragraph{Peter Gustav Lejeune Dirichlet (1805--1859)}
\begin{itemize}
    \item Znaczny wkład w analizę fourierowską
    \item Zdefiniował warunki i ograniczenia konieczne do zbieżności szeregu Fouriera
\end{itemize}

\paragraph{Bernhard Riemann (1826--1866)}
\begin{itemize}
    \item Rozszerzył teorię szeregu Fouriera (rozumienie zbieżności i rozbieżności)
\end{itemize}

\end{document}
